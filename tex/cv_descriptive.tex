\documentclass[10pt]{article}

\usepackage{fullpage}
\usepackage{url}

\pagestyle{empty}

% Customize section headings
%\usepackage{sectsty}
%\sectionfont{\rmfamily\mdseries\Large}
%\subsectionfont{\rmfamily\bfseries\normalsize}

% Don't indent paragraphs.
\setlength\parindent{0em}
\setlength\parskip{0.5em}

% Make lists without bullets
\newenvironment{itemize*}{
   \begin{list}{}
      { 
         \setlength{\itemsep}{5pt}
         \setlength{\parsep}{0pt}
         \setlength{\topsep}{0pt}
         \setlength{\leftmargin}{0em} 
      } 
} {
   \end{list}
}

\begin{document}

{\Large Benjamin R. Hillman} \\
PO Box 5800 \\
Mail Stop 0734 \\
Albuquerque, NM 87185-0734 \\
\url{bhillma@sandia.gov} // (425) 218-8086

\section*{Education}
\begin{itemize*}
    \item Ph.D., Atmospheric Sciences,
    University of Washington, Seattle, WA,
    June 2016
    \item M.S., Atmospheric Sciences,
    University of Washington, Seattle, WA,
    2012
    \item B.S., Physics and Mathematics \textit{Cum Laude},
    Western Washington University, Bellingham, WA,
    2008
    \item A.S.,
    Shoreline Community College, Seattle, WA,
    2005
\end{itemize*}

\section*{Research Experience}
\begin{itemize*}
\item \textbf{Postdoctoral Appointee},
    Department of Atmospheric Science,
    Sandia National Laboratories, Albuquerque, NM,
    Summer 2016--present
    \begin{itemize}
    \item Improving understanding of Arctic cloud processes and model biases through high resolution atmospheric modeling and observations. 
    \item Development and analysis of cutting-edge techniques for improved simulation in global climate models, including the use and development of super-parameterization and regionally-fined meshes. 
    \end{itemize}
\item \textbf{Graduate Research Associate}, 
    Department of Atmospheric Sciences,
    University of Washington, Seattle, WA,
    Fall 2008--Spring 2016 
    \begin{itemize}
    \item Evaluating cloud properties in atmospheric models against satellite remote sensing retrievals using satellite instrument simulators to account for limitations and uncertainties in retrievals. 
    \item Quantification of uncertainties and inherent biases in the satellite simulator framework due to representations of unresolved scales. 
    \item Development and implementation of an improved parameterization of unresolved cloud properties for use in satellite simulators.
   \end{itemize}
\item \textbf{Research Associate},
   Department of Chemistry,
   Western Washington University, Bellingham, WA,
   Summer 2008 
   \begin{itemize}
   \item Modeling growth of thin semiconductor films using a deposition, diffusion, aggregation model.
   \end{itemize}
\end{itemize*}

\section*{Technical Skills}
\begin{itemize}
\item Development and analysis of a range of global climate models, including the GFDL global atmosphere model (AM2), the NCAR Community Earth System Model (CESM), the Super-Parameterized Community Atmosphere Model (SP-CAM), and the DOE Accelerated Climate Model for Energy (ACME)
\item Expertise in the use of satellite instrument simulators for model evaluation
\item Development of analysis tools for end-user applications, including incorporation of new diagnostics into the NCAR Atmosphere Model Working Group (AMWG) diagnostics package
\item Experience with a range of programming and analysis languages including Fortran (77 and 90), C, Python, Matlab, NCL, and UNIX shell scripting
\item Analysis of geospatial datasets using the netCDF operators (NCO)
\item Using git and github for software version control and project management
\item Working in high-performance computing environments
\end{itemize}

\section*{Teaching Experience}
\begin{itemize*}
\item
   Teaching Assistant,
   Atmospheric Radiative Transfer (ATM S 341),
   University of Washington, Seattle, WA,
   Spring 2014
\item 
   Teaching Assistant,
   Introduction to Weather (ATM S 101),
   University of Washington, Seattle, WA,
   Winter 2010
\item
   Teaching Assistant,
   Department of Physics and Astronomy,
   Western Washington University, Bellingham, WA,
   Winter 2006--Spring 2008
\end{itemize*}

\section*{Field Experience}
\begin{itemize*}
   \item Storm Peak Lab Cloud Property Validation Experiment (STORMVEx)
    Steamboat Springs, CO,
    Winter 2011
\end{itemize*}

\section*{Honors}
\begin{itemize*}
   \item 2011 NCAR Advanced Study Program Graduate Visitor
   \item 2008 Dr. James and Joann Albers memorial scholarship
   \item 2007 Dr. James and Joann Albers memorial scholarship
\end{itemize*}

\section*{Publications}
\begin{itemize*}
   \item Hillman, B. R., R. Marchand, T. P. Ackerman, G. G. Mace and S. Benson, 2015: Assessing the accuracy of MISR and MISR-simulated cloud top heights using CloudSat and CALIPSO-retrieved hydrometeor profiles (in review).

   \item Hillman, B. R., R. Marchand, and T. P. Ackerman, 2015: Errors in simulated satellite cloud diagnostics from global climate models due to unresolved cloud structure and variability (in prep).

   \item Hillman, B. R., R. Marchand, and T. P. Ackerman, A. Bodas-Salcedo, J. Cole, J.-C. Golaz, J. E. Kay, 2015: Comparing cloud biases in CMIP5: insights using MISR and ISCCP observations and satellite simulators (in prep).

   \item Hillman, B. R., 2016:
   Reducing errors in simulated satellite views of clouds from
   large-scale models.
   Ph.D. dissertation, University of Washington.

   \item Hillman, B. R., 2012: 
   Evaluating clouds in global climate models using instrument simulators. 
   M.S. thesis, University of Washington.

   \item Kay, J. E., B. R. Hillman, S. A. Klein, Y. Zhang, B. Medeiros,
   R. Pincus, A. Gettelman, B. Eaton, J. Boyle, R. Marchand, 
   and T. P. Ackerman,
   2012:
   Exposing global cloud biases in the Community Atmosphere Model (CAM) using 
   satellite observations and their corresponding instrument simulators. 
   J. Climate, 25, 5190–5207, doi:10.1175/JCLI-D-11-00469.1.
\end{itemize*}

\section*{Selected Presentations}
\begin{itemize*}

   \item Hillman, B. R., R. Marchand, T. P. Ackerman,
   2014:
   Comparison of MISR and MISR-simulated cloud top heights using CloudSat and CALIPSO profiles.
   MISR Science Team Meeting,
   Pasadena, CA.

   \item Hillman, B. R., R. Marchand, T. P. Ackerman, A. Bodas-Salcedo, J. Cole, J.-C. Golaz, J. E. Kay,
   2012:
   Comparing cloud biases in CMIP5: insights using MISR and ISCCP
   American Geophysical Union Fall Meeting, 
   San Francisco, CA.

   \item Hillman, B. R., R. Marchand, T. P. Ackerman, A. Bodas-Salcedo, J. Cole, J.-C. Golaz, J. E. Kay,
   2012:
   An intercomparison of clouds and radiation in CMIP5 models
   using MISR and ISCCP simulators.
   1st Pan-Global Atmosphere Systems Studies (GASS) Conference,
   Boulder, CO.

   \item Hillman, B. R., J. E. Kay, S. A. Klein, Y. Zhang, B. Medeiros,
   R. Pincus, A. Gettelman, B. Eaton, J. Boyle, R. Marchand, and T. P. Ackerman, 
   2011:
   Evaluating clouds in climate models using satellite simulators: 
   from mean state to feedbacks.
   MISR Data Users Symposium, 
   Pasadena, CA.

    \item Hillman, B. R., J. E. Kay, S. A. Klein, Y. Zhang, B. Medeiros,
    R. Pincus, A. Gettelman, B. Eaton, J. Boyle, R. Marchand, and T. P. Ackerman, 
    2011:
    Evaluating clouds in climate models using satellite simulators: 
    from mean state to feedbacks.
    American Geophysical Union Fall Meeting, 
    San Francisco, CA.

    \item Hillman, B.,
    2011:
    Use of satellite instrument simulators in the evaluation of climate models.
    University of Washington Department of Atmospheric Sciences 
    Physics and Chemistry Seminar,
    Seattle, WA.

    \item Hillman, B., J. Kay, and T. Ackerman, 2011:
    Evaluating clouds in the Community Atmosphere Model using COSP.
    Poster presentation, CESM Annual Workshop,
    Breckenridge, CO.

    \item Hillman, B., R. Marchand, and T. Ackerman,
    2010:
    Evaluation of Clouds in Climate Models Using Instrument Simulators.
    Western Washington University Physics Department Invited Colloquium,
    Bellingham, WA.

    \item Hillman, B., R. Marchand, and T. Ackerman, 
    2010:
    Evaluation of Low Clouds in the NCAR CAM3 and GFDL AM2 Using MISR Joint 
    Histograms.
    American Geophysical Union Fall Meeting, 
    San Francisco, CA.

    \item Hillman, B., R. Marchand, and T. Ackerman,
    2010:
    Evaluation of Low Clouds in the NCAR CAM3 and GFDL AM2 Using MISR Joint 
    Histograms.
    MISR Data Users Symposium, 
    Pasadena, CA.

    \item Hillman, B., T. Ackerman, and R. Marchand,
    2009:
    Evaluating global climate models using a MISR simulator.
    Presentation, 
    MISR Data Users Science Symposium, 
    Pasadena, CA.

\end{itemize*}

\end{document}
